\documentclass[11pt, a4paper]{article}
\usepackage[left=25mm, right=25mm, top=20mm, bottom=20mm, includefoot]{geometry}
\usepackage{setspace}
\usepackage{titlesec}
\usepackage{graphicx}
\usepackage[english]{babel}
\usepackage[autostyle]{csquotes}
\usepackage{xcolor}
\usepackage{listings}
\usepackage{array}
\usepackage[toc, title, page]{appendix}
\usepackage{pdfpages}
\usepackage[style=apa, backend=biber, language=english]{biblatex}
\PassOptionsToPackage{hyphens}{url}\usepackage{hyperref}

\addbibresource{sources.bib}
\setlength{\parindent}{0ex}
\setlength{\parskip}{1.5ex}
\font\titlefont=cmr10 at 18pt
\MakeOuterQuote{"}

\titleformat{\chapter}[display]
  {\normalfont\huge\bfseries\flushleft\hyphenpenalty=10000}{\chaptertitlename\ \thechapter}{20pt}{\Huge}
\titlespacing*{\chapter}{0pt}{0pt}{40pt}

\lstset{
  frame=tb, % draw a frame at the top and bottom of the code block
  tabsize=4, % tab space width
  showstringspaces=false, % don't mark spaces in strings
  numbers=left, % display line numbers on the left
  basicstyle=\linespread{0.85}\selectfont,
  breaklines=true,
  postbreak=\mbox{\textcolor{red}{$\hookrightarrow$}\space},
  commentstyle=\color{olive}, % comment color
  keywordstyle=\color{blue}, % keyword color
  stringstyle=\color{red} % string color
}

\linespread{1.5}
\begin{document}

\title{\titlefont 4D Geometry rendering assignment}
\author{Ashley Smith}
\date{\today}
\maketitle
\thispagestyle{empty}

% Abstract goes here if desired

\cleardoublepage
\setcounter{page}{1}
\pagenumbering{roman}
\tableofcontents

\cleardoublepage
\pagenumbering{arabic}
\setcounter{page}{1}

\section{Background description (10\%)}

For this assignment, I have chosen to learn about and investigate the fourth dimension with respect to graphics. Most games are typically set in a two or three dimensional world and the fourth dimension is rarely discussed as part of a game's core design, if discussed at all. I aim to create an interactive visualisation of four-dimensional geometry in an effort to educate both myself and others how the fourth dimension may be applicable to graphics used in video games or other media outlets.

\subsection{2D to 3D: Why do dimensions matter?}

A game's graphics, character controls, world concepts and designs and even a game's genre are affected by the number of dimensions a game has -- this can be seen in \citeauthor*{riskofrain}' Risk of Rain \parencite*{riskofrain} and its sequel Risk of Rain 2 \parencite{riskofrain2}. In two-dimensional games, characters are generally confined to two axes and are therefore considered to be top-down or side-on; Risk of Rain is a platformer and makes use of a side-on perspective. Because of this, enemies line up and then stack on top of each other to attack the player, who's only choices of evasion are to run away or jump over the enemies. Risk of Rain 2 however is 3D, meaning that the enemies attempt to surround the player and the player can escape in multiple directions while also retaining their option to try and jump over enemies.

\emph{TODO: RISK OF RAIN IMAGE HERE}

It could be said that the transition to 3D was good for Risk of Rain \parencite*{riskofrain}, however, there are many features in the first game that couldn't be brought into the game. Some player abilities that shot in straight lines to hit multiple enemies wouldn't be as effective if the enemies weren't lined up or stacked on top of one another; ladders and other platforming elements that took advantage of the space on a player's screen would look awkward and cramped in an otherwise-spacious 3D environment, meaning that the player is mainly stuck running around on the ground; and finally, while the player can see more things in front of them without the constraint of one's monitor size, enemies can easily spawn or wander behind the player and perform surprise (or sometimes \emph{unfair}) attacks and ending the player's game. Despite these shortcomings, Risk of Rain 2 \parencite*{riskofrain2} is considered a well-received improvement to the original \parencite{riskofrain2steamreviews}.

\subsection{3D to 4D: What will be achieved from this project?}

If the third dimension can create new opportunities and conceptualise new gameplay and new mechanics, it does make one wonder about the possibilities that the fourth dimension may bring to gaming. With virtual reality and other technologies becoming more accessible and accepted as avenues of entertainment, the fourth dimension may not seem so experimental. In this project, I will be attempting to write my own 4D geometry viewer in an effort to lessen the gap between the third and fourth dimensions. While this project doesn't feature anything new, it is hoped that the development of this project is just as informative as the end result and that the approaches used to build this application will be of use to future developers wanting to pioneer the fourth dimension.

Visualising the fourth dimension in simulations and games isn't simple. 2D games can be seen as simplified 3D worlds, especially when art and rendering techniques are used to try and mimic the 3D world as seen in classic arcade games like After Burner \parencite{afterburner}. Humans are three-dimensional beings and as such we feel familiar with worlds that appear to be 3D even when looking at them through a two-dimensional display. When a 3D game is rendered, matrix mathematics is employed to translate coordinates in world space to screen space and therefore render a 3D world onto the screen. 4D objects and worlds are incomprehensible to beings who can only perceive the third dimension, meaning that there is always going to be a sense of esoterism when working with anything beyond three dimensions. Moreover, 4D geometry needs to be translated into 2D geometry in order to be rendered on screen, creating an additional barrier to visualisation.

END

- Describe chosen area with strong rationale for why its important to investigate
- Describes the scope and purpose of deliverable and what I'm expecting to achieve
- Describe the level of difficulty along with relevance of work to field of graphics / games

\section{Standout elements (10\%)}

- Describe standout elements (things considered important or critical, areas I'm proudest of, areas I struggled but then succeeded with)
- Provide detail as to why they were important or how they relate to deliverable
- Provide details, insight and rationale about the above

\section{Design (30\%)}

- Describe how the deliverable works in detail
- Use appropriate and recognised notations and abstraction mechanisms
- Clear and detailed evidence of the use of software engineering principles (software architecture)
- Detail the graphical techniques or concepts employed and how they were implemented
- Describe any algorithms that relate to delivarable

\section{Implementation (10\%)}

- Describe the process and approach of development along with any milestones
- Comment on what happen in each step, what was overcome and what was learnt
- Describe what steps would be useful in future projects and which ones wouldn't

\cleardoublepage
\printbibliography[
  heading=bibintoc,
  title={Bibliography}
]

\cleardoublepage
\begin{appendices}

\section{Something}

It's an appendix

\end{appendices}

\end{document}
